\chapter*{ВВЕДЕНИЕ}
\addcontentsline{toc}{chapter}{ВВЕДЕНИЕ}

Технологии компьютерного зрения и искусственного интеллекта находят применение в разных сферах человеческой деятельности.
Важным и интересным направлением, где возможно применение данных технологий, является анализ объектов на медицинских изображениях.
На сегодняшний день анализ МИ и поиск объектов на них широко применяется в медицинской диагностике~-- от анализа крови до магнитнорезонансной томографии.
До недавнего времени задачи анализа МИ решались с использованием различных алгоритмов, основанных на использовании гистограмм градиентов, алгоритмов каскадных классификаторов на основе метода Виолы~---~Джонса, алгоритмов, основанных на методах контурного анализа и др.
Традиционные методы анализа МИ и поиска на них объектов достигли своего предела производительности.
Аналогично медицинской сфере, подход распознавания объектов с использованием нейронных сетей нашел свое применение и в задачах мониторинга морского дна~\cite{intro1, intro2}.

Для решения задачи распознавания объектов зачастую выбирают сверточные нейронные сети из-за простоты реализации, минимальных системных требований и хорошего процента распознавания объектов. 
Сверточная нейронная сеть (CNN)~--- частный случай искусственных нейронных сетей глубокого обучения.
Архитектура сверточных сетей была предложена Яном Лекуном в 1988 году с целью повышения эффективности распознавания образов~\cite{cnn, cnndef}.

Целью работы является сравнение алгоритмов поиска объектов на изображениях с использованием различных модификаций сверточных нейронных сетей.
Для достижения поставленной цели необходимо решить следующие задачи:
\begin{enumerate}
    \item провести анализ предметной области алгоритмов поиска объектов на изображениях;
    \item описать основные подходы к решению задачи распознавания объектов на изображениях;
    \item сформулировать критерии сравнения применяемых методов и выполнить их сравнение.
\end{enumerate}