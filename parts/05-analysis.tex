\chapter{Описание предметной области}

\section{Задача поиска объекта на изображении}

Задача поиска объекта на изображении сводится к решению следующих подзадач~\cite{task}: 
\begin{enumerate}
	\item сегментация~--- выделение участков изображения, которые относятся к разным объектам;
	\item классификация~--- определение типа объекта, для каждого выделенного сегмента отдельно.
\end{enumerate}
Таким образом, обнаружение объектов~--- это процесс сегментации и классификации объектов в изображении. 

\section{Сверточные нейронные сети для поиска объектов на изображениях}

Сверточные нейронные сети являются наиболее распространенным алгоритмом глубокого обучения, применяющим несколько сверхточных слоев и вычислений.
Они предоставляют эффективные способы извлечения признаков, а также являются лучшим выбором для решения проблем обнаружения объектов.
Текущие подходы с использованием методов глубокого обучения для задач классификации и регрессии объектов можно разделить на две категории:
\begin{enumerate}
	\item двухэтапные методы, которые представлены такими архитектурами как R-CNN, FastR-CNN и FasterR-CNN;
	\item одноэтапные алгоритмы, представленные различными версиями YOLO и др.
\end{enumerate}
В двухэтапных методах используется селективный поиск или сеть региональных предположений (англ. RPN) для выделения областей, с высокой вероятностью содержащих внутри себя объекты.
Затем при помощи классификатора определяется класс объекта, а при помощи регрессора определяются ограничивающие рамки.
Данный метод обладает высокой точностью, но при этом ограничен в скорости обнаружения.
Одноэтапные алгоритмы не используют отдельную сеть для генерации регионов и основываются на методах регрессии, просматривая изображения целиком.
Так как данные алгоритмы не используют RPN, скорость обнаружения выше, но точность выделения, в особенности малых объектов, не такая высокая, как у двухэтапных методов~\cite{base}.