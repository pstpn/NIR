\chapter{Описание предметной области}

\section{Оптимизационная задача}

Постановка каждой задачи оптимизации (ЗО) включает в себя два объекта: 
\begin{itemize}
	\item множество допустимых решений;
	\item целевую функцию, которую следует минимизировать или максимизировать на указанном множестве~\cite{metodoptimization}.
\end{itemize}

Результатом решения ОЗ являются наилучшие, в некотором смысле, структура и значения параметров системы. 
Определение оптимальных значений параметров системы при заданной ее структуре называется параметрической оптимизацией~\cite{paramoptimization}.

Задача оптимизации в целом сводится к задаче поиска экстремума (минимума или максимума) целевой функции. 
Заданы множество $X$ и функция $f(x)$, определенная на $X$. Требуется найти точки минимума или максимума, в зависимости от условий задачи:
\begin{equation}
	\begin{gathered}
		f(x) \to (\min, \max), \forall x \in X,
	\end{gathered}.
\end{equation}
где $f(x)$~--- целевая функция;
$X$~--- допустимое множество.

В многокритериальной задаче оптимизации сравнение решений по предпочтительности осуществляется не непосредственно, а при помощи заданных на $X$ числовых функций $f_1, f_2, \dots ,f_m$, называемых критериями. 
Предполагается, что $m \leq 2$: при $m = 1$ задача оптимизации является однокритериальной.

\clearpage
\section{Генетические алгоритмы}
Адаптивные методы основаны на генетических процессах в биологических организмах: биологические популяции развиваются в течение нескольких поколений, подчиняясь законам естественного отбора и принципу «выживает наиболее приспособленный», открытому Чарльзом Дарвиным~\cite{GA}.

Основными операторами ГА являются: кроссинговер, мутация, выбор родителей и селекция~\cite{GA}. 

Основные этапы работы ГА:
\begin{enumerate}
	\item генерируем начальную популяцию из $N$ хромосом;
	\item вычисляем для каждой хромосомы ее пригодность;
	\item выбираем пару хромосом--родителей с помощью одного из способов отбора;
	\item проводим кроссинговер двух родителей с вероятность $P_c$, производя двух потомков;
	\item проводим мутацию потомков с вероятностью $P_m$;
	\item повторяем шаги 3-5, пока не будет сгенерировано новое поколение популяции, содержащее $N$ хромосом;
	\item повторяем шаги 2-6, пока не будет достигнут критерий окончания процесса.
\end{enumerate}

На рисунке~\ref{figure:mashin01} представлена схема вышеописанного алгоритма. 
\begin{figure}[H]
	\centering
	\includegraphics[width=\textwidth, height=0.6\textheight, keepaspectratio]{img/algorithm_work.pdf}
	\caption{Схема работы алгоритма}
	\label{figure:mashin01}
\end{figure}


