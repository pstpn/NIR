\chapter{Сравнение модификаций CNN при решении задач поиска объектов на изображении}

\section{Результаты сравнения}

Для сравнения модификаций сверточных нейронных сетей будут использоваться следующие основные критерии~\cite{base, all, yolobase, cmp}:
\begin{enumerate}
	\item mAP;
	\item FPS;
	\item PS.
\end{enumerate}

Критерий mAP показывает точность модели и вычисляется по формуле~\ref{eq2}.
\begin{equation}
	\label{eq2}
	mAP = 1/N \sum_{i=1}^{N} AP_{i}
\end{equation}
, где N~---~ количество классов,
AP (average precision)~---~популярная метрика для измерения точности детекторов объектов.

Критерий FPS показывает, сколько кадров в секунду может обрабатывать нейронная сеть.

Критерий PS показывает общую оценку каждой системы распознавания, основываясь на том, что точность и скорость имеют одинаковый вес и вычисляется по формуле~\ref{eq3}.
\begin{equation}
	\label{eq3}
	PS = mAP \times FPS
\end{equation}

В таблице~\ref{table1} представлены количественные результаты сравнения различных модификаций сверточных нейронных сетей на основе сформированных критериев~\cite{all, yolobase}.

\clearpage
\begin{table}[!ht]
	\centering
	\caption{\label{table1} Количественные результаты сравнения различных модификаций CNN на основе сформированных критериев}
	\begin{tabularx}{\textwidth}{|X|c|c|c|}
		\hline
		Модификация CNN & mAP & FPS & PS \\ \hline
		R-CNN Minus R & 53.5 & 6 & 321.0 \\ \hline
		Fast R-CNN & 70.0 & 0.5 & 35.0 \\ \hline
		Faster R-CNN VGG-16 & 73.2 & 7 & 512.4 \\ \hline
		Faster R-CNN ResNet & 76.4 & 5 & 382.0 \\ \hline
		YOLOv1 & 63.4 & 45 & 2853.0 \\ \hline
		YOLOv2 288 $\times$ 288 & 69.0 & \textbf{91} & \textbf{6279.0} \\ \hline
		YOLOv2 352 $\times$ 352 & 73.7 & 81 & 5970.0 \\ \hline
		YOLOv2 416 $\times$ 416 & 76.8 & 67 & 5146.0 \\ \hline
		YOLOv2 480 $\times$ 480 & 77.8 & 59 & 4590.2 \\ \hline
		YOLOv2 544 $\times$ 544 & \textbf{78.6} & 40 & 3144.0 \\ \hline
	\end{tabularx}
\end{table}

\section{Вывод}

Исходя из полученной сравнительной таблицы~\ref{table1}, можно выделить следующие пункты:
\begin{enumerate}
	\item по двум из трех сформированных критериев лидирует модель YOLOv2 288 $\times$ 288, обладающая наибольшим значением FPS и PS;
	\item наибольшей точностью обладает модель YOLOv2 544 $\times$ 544 (по критерию mAP);
\end{enumerate}

Таким образом, можно сделать вывод, что использование класса моделей YOLO может применяться в задачах поиска изображений в реальном времени, так как именно эти модификации CNN обладают наибольшим значением FPS (от 40 до 91).
Модели класса Faster R-CNN имеют аналогичные значения mAP в сравнении с моделями YOLO, однако скорость обработки изображений меньше $\sim$~в  7~---~9 раз относительно модели YOLOv2 544 $\times$ 544, и в $\sim$~в  13~---~18 раз относительно модели YOLOv2 288 $\times$ 288.
Сеть Fast R-CNN обладает наименьшей скоростью поиска объектов (FPS) среди всех представленных модификаций CNN, а самая базовая модель R-CNN Minus R имеет наименьший показатель точности (mAP).