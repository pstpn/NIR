\chapter{Анализ генетических алгоритмов при решении задач многокритериальной оптимизации}

\section{Описание задачи}
Целью задачи распределения рабочих процессов в процессе производства является эффективное распределение рабочих процессов на доступные машины для минимизации общего времени производства. 
У каждого рабочего процесса есть определенное время выполнения и заданы ограничения на количество рабочих процессов, которые можно выполнять одновременно на каждой машине.
Задача многокритериальной оптимизации заключается в поиске оптимального распределения рабочих процессов, учитывая не только время выполнения, но и количество используемых машин.
\begin{equation}
	\begin{gathered}
		f(x) \to (\min, \max), \forall x \in X
	\end{gathered}.
\end{equation}
где $f(x)$~--- целевая функция;
$X$~--- допустимое множество.


\section{Способы решения задачи}

\begin{enumerate}
	\item Простой генетический алгоритм:
	В данном случае, исходя из заданных ограничений и целевых функций, можно определить особи в виде хромосом, представляющих возможные распределения рабочих процессов на машины. 
	Затем, с помощью операций скрещивания и мутации, можно создать новые поколения особей, оценивая их по целевым функциям и отбирая лучших особей для следующего поколения. 
	Таким образом, простой генетический алгоритм будет итеративно искать оптимальное распределение.
	\item Гибридный генетический алгоритм: 
	для данной задачи можно использовать гибридный подход, комбинируя генетический алгоритм с другими методами оптимизации, такими как локальный поиск или эволюционные стратегии. 
	Например, можно использовать генетический алгоритм для глобального поиска оптимального решения, а затем применить локальный поиск для улучшения найденного решения.
	\item Параллельный и распределенный генетический алгоритм: 
	для данной задачи, где решение требует обработки большого объема данных и выполнения вычислительно сложных операций, можно применить параллельные и распределенные вычисления. 
	Несколько генетических алгоритмов могут работать параллельно, обмениваясь информацией и объединяя свои результаты для получения более точного и быстрого решения.
	\item Адаптивный генетический алгоритм: 
	для данной задачи, где условия и ограничения могут изменяться со временем, можно применить адаптивный генетический алгоритм. 
	В этом случае, генетический алгоритм будет способен адаптироваться к изменяющимся условиям и динамически настраивать свои параметры, такие как вероятность мутации или скрещивания, чтобы достичь оптимального решения.
\end{enumerate}

\section{Выводы}

\begin{table}[h!]
	\begin{center}
		\caption{\label{table:models} Сравнительная таблица для методов описания объектов на сцене}
		\begin{tabular}{|p{85pt}|p{85pt}|p{85pt}|p{85pt}|p{85pt}|}
			\hline
			~ & \textbf{AGA} & \textbf{HGA} & \textbf{SGA} & \textbf{PGA} \\
			\hline
			Адаптивность & Изменяет параметры ГА во время выполнения в зависимости от эффективности & Использует комбинацию нескольких методов оптимизации таких как ГА с другими алгоритмами & Фикс. параметры ГА в течение всего процесса оптимизации & Может использовать несколько экземпляров ГА работающих параллельно \\ \hline
			Гибридизация & Может быть гибридизирован с другими методами оптимизации для повышения эффективности & Интегрирует в себя элементы нескольких различных алгоритмов оптимизации & Основан исключительно на принципах генетического программирования & Может использовать комбинацию различных ГА для решения задачи \\ \hline
			Простота реализации & Требует дополнительных усилий для настройки адаптивных механизмов & Может потребовать сложной интеграции различных алгоритмов & Прост в реализации и понимании, так как использует базовые принципы ГА & Реализация может потребовать учета аспектов параллельного программирования \\ \hline
			Склонность к застреванию в локальных оптимумах & Минимиз. за счет адаптивности, & Зависит от конкретной гибридизации, которая может изменять параметр & Может сталкиваться с проблемой застревания в локальных оптимумах из-за отсутствия механизмов адаптивности & Может использовать параллельные  процессы для более широкого поиска в пространстве решений \\  \hline
		\end{tabular}
	\end{center}
\end{table}


